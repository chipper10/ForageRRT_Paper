%&pdflatex
\documentclass[conference]{IEEEtran}
\usepackage{graphicx}
\begin{document}
\title{Forage RRT - An Efficient Approach To Task-Space Goal Planning for Redundant Manipulators}
\maketitle
\section{Introduction}
One of the ultimate goals of motion planning algorithms for manipulators is to allow the manipulator to automatically find a collision-free
path between any given start configuration and a goal point specified in end-effector space coordinates. The goal is ideally specified in
end effector space coordinates because it is almost always the end effector which interacts with the object to be manipulated. The specific
configuration of the other links of the manipulator is usually not important as long as it is not in collision with workspace obstacles.
Examples of end effector tools which interact with objects include hands, magnets, suction cups, paint sprayers, and welders. Thus, any
practical manipulator planner has to convert end-effector space goal coordinates to configuration space goal coordinates, i.e. the
configuration of the manipulator which achieves the given end effector goal coordinate.

In the literature, the two main approaches to solving this problem have been to either figure out the configuration-space goal coordinate
directly using inverse kinematics or to incorporate the search for the configuration space goal coordinate into the planning. To date, we
are not aware of any inverse kinematics algorithm for general n-link manipulators which is complete, fast, and returns a configuration
guaranteed to be reachable from the start configuration. Incorporating inverse kinematics into the planning, especially when the planning
algorithm is the Rapidly-Exploring Random Tree (RRT) alleviates all of these problems. However, this approach struggles to efficiently solve
a certain class of problems, known as bug-trap problems, wherein the approach to the start or goal is largely occluded by an obstacle. If
the configuration of the goal were known, this problem can be satisfactorily solved by the Bidirectional RRT. 

In this paper, we present the Forage RRT which searches for the goal configuration as part of planning, making it complete, but also tackles
bug trap type problems with relative ease. The result is a reliable planner which is fast and consistent at solving a wide range of
manipulation problems in environments with obstacles. The main idea behind the approach is to initially explore the manipulator space
quickly with a large step-size RRT and then attempt to connect to goal using a small step-size Jt-RRT from promising nodes in the large
step-size RRT. We believe the ease of implementing this approach along with its excellent performance on all problems will allow it to be
used as a general manipulation planner both in industry and academia.

The layout of this paper will be to present previous work which we build upon as well as other approaches to the same problem, an analysis
of the shortcoming of the RRT in bug trap problems, the implementation of the Forage-RRT algorithm, experiments and results compared to
other planners having the same problem statement, and finally a discussion of possible improvement to the Forage RRT.   

\section{Related Work}
 
\section{RRT and the Bug-Trap Problem}

\section{Forage RRT Implementation}

\section{Experiments and Evaluation}

\section{Discussion}

\bibliographystyle{plain}
\bibliography{RRT}
\bibliography{MotionPlanning}
\bibliography{IK}

\end{document}

